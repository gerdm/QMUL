\documentclass[11pt]{article}
\usepackage[utf8]{inputenc}
\usepackage{amsmath}
\usepackage{amssymb}
\usepackage{amsthm}

\newcommand{\argmax}[1]{\underset{#1}{\operatorname{arg}\,\operatorname{max}}\;}
\newtheorem{proposition}{Proposition}[section]

\title{Kalman Filters: an introduction}
\author{Gerardo Durán Martín}
\begin{document}
\maketitle

\section{Introduction}

Dynamical systems are models that capture the time evolution of a system. A discrete-time linear dynamical system is a dynamical system in which an $M$-dimensional vector ${\bf x}_n$ evolves in the form.

\begin{equation}
	{\bf x}_{n+1} = {\bf A x}_n
\end{equation}

An example of a linear dynamical system is...


Suppose $\{{\bf z}_t\}_t$ is a signal coming from an unknown source and $\{{\bf x_t}\}$ are observed values derived from ${\bf z}_t$. If the system is linear and deterministic we write the equations that govern such system in the form

\begin{align*}
	{\bf z}_{n+1} &= {\bf A} {\bf z}_{n}\\
	{\bf x}_{n+1} &= {\bf C} {\bf z}_{n+1}
\end{align*}

In this model, we assume the existence of an unobserved signal ${\bf z}_t$ that evolves over time according to the matrix $\bf A$. In some settings, however, it is not the case that the signal we are trying to find behaves deterministically. To take into account the uncertainty of the system, we assume that that both the underlying signal and the observed signal are corrupted by some noise. Under this scenario, we modify our previous system as follows:
 

\begin{align*}
	{\bf z}_{n+1} &= {\bf A} {\bf z}_{n} + \boldsymbol\varepsilon_n\\
	{\bf x}_{n+1} &= {\bf C} {\bf z}_{n+1} + \boldsymbol\varphi_n
\end{align*}

The noise terms $\boldsymbol{\varepsilon}$, and $\boldsymbol{\varphi}$ are assumed to be normally distributed with mean zero and covariance matrices $\boldsymbol{\Gamma}$, and $\boldsymbol{\Sigma}$.



\begin{align}
	\boldsymbol\varepsilon_n &\sim \mathcal{N}({\bf 0}, \boldsymbol\Gamma)\\
	\boldsymbol\varphi & \sim \mathcal{N}({\bf 0}, \boldsymbol\Sigma)
\end{align}

With our specified assumptions, the likelihood of the model is given by

\begin{equation}
	p({\bf X}, {\bf Z}\vert\boldsymbol\theta) = p({\bf z}_1)\prod_{n=2}^N p({\bf z}_n\vert {\bf z}_{n-1})\prod_{n=1}^N p({\bf x}_n\vert {\bf z}_n)
\end{equation}

Our purpose is to find parameters $\boldsymbol\theta = \{{\bf A}, {\bf C}, \boldsymbol\Gamma, \boldsymbol\Sigma\}$ that best represent the data. This is usually done maximising the likelihood of the data with respect to the parameters $\boldsymbol{\theta}$. Maximising the likelihood of this model, however, is not straightforward. This is because the only observed data we have is given by the dataset $\{{\bf x}_n\}_n$. % Explain more this step


One approach of finding such parameters is to make use of the EM algorithm: an iterative approach to maximise the likelihood of the complete-data log-likelihood as follows

\begin{align}
	\boldsymbol{\theta}^\text{new} &= \argmax{\boldsymbol{\theta}} \mathbb{E}_{{\bf Z}\vert {\bf X}, \boldsymbol{\theta}^\text{old}}[\log p({\bf X}, {\bf Z}\vert \boldsymbol\theta)]\\
	&= \argmax{\boldsymbol{\theta}} Q(\boldsymbol\theta, \boldsymbol\theta^\text{old})
\end{align}

Where we have defined $Q(\boldsymbol\theta, \boldsymbol\theta^\text{old}) := \mathbb{E}_{{\bf Z}\vert {\bf X}, \boldsymbol{\theta}^\text{old}}[\log p({\bf X}, {\bf Z}\vert \boldsymbol\theta)]$.The computation of the posterior probability of latent variables $p({\bf Z}\vert {\bf X}, \boldsymbol\theta)$ is called the E-step. The maximisation of the complete-data log-likelihood with respect to the posterior latent variables is called the M-step.


\subsection{The E-step}
To make use of the E-step, it is helpful to distinguish which elements depend on ${\bf z}_n$. Consider

\begin{align}
	\log p({\bf X}, {\bf Z}\vert \boldsymbol\theta) &= \log p({\bf z}_1) + \sum_{n=2}^N \log p({\bf z}_n\vert {\bf z}_{n-1}) + \sum_{n=1}^N \log p({\bf x}_n\vert {\bf z}_n)\\
	   &=\frac{1}{2}\log\vert{\bf V}_0^{-1}\vert - \frac{1}{2}({\bf z}_1 - \boldsymbol\mu_0)^T{\bf V}_0^{-1}({\bf z}_1 - \boldsymbol\mu_0) \nonumber \\
	   &\hspace{1cm}+ \sum_{n=2}^N \frac{1}{2}\log\vert\boldsymbol\Gamma^{-1}\vert - \frac{1}{2}({\bf z}_n - {\bf A z}_{n-1})^T\boldsymbol{\Gamma}^{-1}({\bf z}_n - {\bf A z}_{n-1}) \nonumber \\
	   &\hspace{1cm}+\sum_{n=1}^N \frac{1}{2}\log\vert\boldsymbol\Sigma\vert -\frac{1}{2}({\bf x}_n - {\bf C z}_n)^T \boldsymbol{\Sigma}^{-1}({\bf x}_n - {\bf C z}_n) + \text{const.}\\
	   &= \frac{1}{2}\log \vert
	  {\bf V}_0^{-1}\vert -\frac{1}{2}\left[\text{Tr}\left({\bf z}_1 {\bf z}_1^T {\bf V}_0^{-1}\right) -2 {\bf z}_1{\bf V}_0^{-1}\boldsymbol\mu_0 + \boldsymbol\mu_0 {\bf V}_0^{-1}\boldsymbol\mu_0^T\right] \nonumber \\
	  &\hspace{1cm}+ \frac{N-1}{2}\log\vert\boldsymbol\Gamma^{-1}\vert + \frac{N}{2}\log\vert \boldsymbol\Sigma^{-1}\vert \nonumber \\
	  &\hspace{1cm}-\frac{1}{2} \sum_{n=2}^{N}\text{Tr}\left({\bf z}_n{\bf z}_n^T\boldsymbol\Gamma^{-1} -2 {\bf z}_{n-1}{\bf z}_n^T{\bf A}\boldsymbol\Gamma^{-1} +  {\bf z}_{n-1}{\bf z}_{n-1}^T{\bf A}\boldsymbol\Gamma^{-1}{\bf A}\right) \nonumber\\
	  &\hspace{1cm}-\frac{1}{2}\sum_{n=1}^N\left[{\bf x}_n^T\boldsymbol{\Sigma}^{-1}{\bf x}_n-2{\bf z}_n^T\boldsymbol\Sigma^{-1}{\bf x}_n + \text{Tr}({\bf z}_n{\bf z}_n^T{\bf C}\boldsymbol\Sigma^{-1}{\bf C})\right] \nonumber\\
	  &\hspace{1cm} + \text{const.}
\end{align}

Where const. are the terms that do not depend on $\boldsymbol{\theta}$. From this last equation, we note that the expectation with respect to the posterior distribution comes only in the form of the expectations $\mathbb{E}[{\bf z}_n]$, $\mathbb{E}[{\bf z}_n{\bf z}_{n}^{T}]$, and  $\mathbb{E}[{\bf z}_n{\bf z}_{n-1}^{T}]$. We obtain:

\begin{align}
	Q(\boldsymbol\theta, \boldsymbol\theta^\text{old}) &= \frac{1}{2}\log \vert
	  {\bf V}_0^{-1}\vert -\frac{1}{2}\left[\text{Tr}\left(\mathbb{E}\left[{\bf z}_1 {\bf z}_1^T\right] {\bf V}_0^{-1}\right) -2 \mathbb{E}\left[{\bf z}_1\right]{\bf V}_0^{-1}\boldsymbol\mu_0 + \boldsymbol\mu_0 {\bf V}_0^{-1}\boldsymbol\mu_0^T\right] \nonumber \\
	  &\hspace{1cm}+ \frac{N-1}{2}\log\vert\boldsymbol\Gamma^{-1}\vert + \frac{N}{2}\log\vert \boldsymbol\Sigma^{-1}\vert \nonumber \\
	  &\hspace{1cm}-\frac{1}{2} \sum_{n=2}^{N}\text{Tr}\left(\mathbb{E}\left[{\bf z}_n{\bf z}_n^T\right]\boldsymbol\Gamma^{-1} -2\mathbb{E}\left[ {\bf z}_{n-1}{\bf z}_n^T\right]{\bf A}\boldsymbol\Gamma^{-1} + \mathbb{E}\left[{\bf z}_{n-1}{\bf z}_{n-1}^T\right]{\bf A}\boldsymbol\Gamma^{-1}{\bf A}\right) \nonumber\\
	  &\hspace{1cm}-\frac{1}{2}\sum_{n=1}^N\left[{\bf x}_n^T\boldsymbol{\Sigma}^{-1}{\bf x}_n-2\mathbb{E}\left[{\bf z}_n^T\right]\boldsymbol\Sigma^{-1}{\bf x}_n + \text{Tr}(\mathbb{E}\left[{\bf z}_n{\bf z}_n^T\right]{\bf C}\boldsymbol\Sigma^{-1}{\bf C})\right] \nonumber\\
	  &\hspace{1cm} + \text{const.}
\end{align}

Before computing the expected values of the latent variables, it is useful to find the posterior distributions of the latent variables $\gamma({\bf z}_n) := p({\bf z}_n\vert {\bf X})$, and $\xi({\bf z}_{n-1}, {\bf z}_{n}) := p({\bf z}_{n-1}, {\bf z}_n\vert {\bf X})$. %why is it useful?

\begin{proposition}
	The term $\gamma({\bf z}_n)$ can be written as a product of the joint probabilities of the first $n$ observations and ${\bf z}_n$, and the conditional probability of the observations that follow ${\bf z}_n$ conditional on ${\bf z}_n$.
\end{proposition}

\begin{proof}\label{prop:gamma-factorisation}
Consider  $\gamma({\bf z}_n)$ and equation \ref{gm:1} from proposition \ref{prop:graphical-models-separation}. We have
\begin{align}
	\gamma({\bf z}_n) &= p({\bf z}_n \vert {\bf X})\\
					  &= \frac{1}{p({\bf X})}p({\bf z}_n)p({\bf X} \vert {\bf z}_n)\\
					  &= \frac{1}{p({\bf X})} p({\bf z}_n) p({\bf x}_1, \ldots, {\bf x}_n\vert {\bf z}_n) p({\bf x}_{n+1}, \ldots, {\bf x}_N\vert {\bf z}_n)\\
					  &= \frac{1}{p({\bf X})}p({\bf x}_1, \ldots, {\bf x}_n, {\bf z}_n) p({\bf x}_{n+1}, \ldots, {\bf x}_N\vert {\bf z}_n)\\
					  &= \frac{1}{p({\bf X})}\alpha({\bf z}_n)\beta({\bf z}_n)
\end{align}

Where we have defined $\alpha({\bf z}_n) := p({\bf x}_1, \ldots, {\bf x}_n, {\bf z}_n)$ as the joint probability of the observed data up to $n$, and $\beta({\bf z}_n) := p({\bf x}_{n+1}, \ldots, {\bf x}_N\vert {\bf z}_n)$ as the probability of all data that follows the $n$-th observation, conditional on the $n$-th latent variable ${\bf z}_n$.
\end{proof}


% Here we show that the xi can also be written in terms of alpha and beta
\begin{proposition}\label{prop:xi-factorisation}
	The term $\xi({\bf z}_{n-1}, {\bf z}_n)$ can be written in terms of $\alpha({\cdot})$, and $\beta(\cdot)$
\end{proposition}

\begin{proof}
	\begin{align}
		\xi({\bf z}_{n-1}, {\bf z}_{n}) &= p({\bf z}_{n-1}, {\bf z}_{n} \vert {\bf X})\\
		&= \frac{1}{p({\bf X})}p({\bf z}_{n-1}, {\bf z}_{n}) p({\bf X} \vert {\bf z}_{n-1}, {\bf z}_{n})\\
		&= \frac{1}{p({\bf X})} p({\bf z}_{n-1}) p({\bf z}_{n} \vert {\bf z}_{n-1}) p({\bf x}_1, \ldots, {\bf x}_{n-1}\vert {\bf z}_{n-1}) \nonumber \\
			&\hspace{1cm} p({\bf x}_n\vert {\bf z}_n) p({\bf x}_{n+1}, \ldots, {\bf x}_N \vert {\bf z}_n)\\
		&= \frac{1}{p({\bf X})} p({\bf z}_{n} \vert {\bf z}_{n-1}) p({\bf x}_1, \ldots, {\bf x}_{n-1}, {\bf z}_{n-1}) \nonumber \\
			&\hspace{1cm} p({\bf x}_n\vert {\bf z}_n) p({\bf x}_{n+1}, \ldots, {\bf x}_N \vert {\bf z}_n)\\
		&= \frac{1}{p({\bf X})} \alpha({\bf z}_{n-1}) p({\bf z}_{n} \vert {\bf z}_{n-1}) p({\bf x}_n \vert {\bf z}_n) \beta({\bf z}_n)
	\end{align}
\end{proof}


% Hence, to make sense of gamma and xi we only need to find values of alpha and beta
Propositions \ref{prop:gamma-factorisation}, and \ref{prop:xi-factorisation} show that both $\gamma$ and $\xi$ are completely determined by the values of $\alpha$, and $\beta$. Hence, we only need to compute the set of values $\{\alpha({\bf z}_n)\}_n$ and $\{\beta({\bf z}_n)\}_n$ once per E-iteration to find the distributions of $\gamma$ and $\xi$. Our next step is to derive an algorithm to efficiently compute $\alpha({\bf z}_n)$, and $\beta({\bf z}_n)$ for every $n$.

% 	* We show that alpha can be represented as a recursive formula
\begin{proposition}
	$\alpha({\bf z}_n)$ can be written recursively as
	\begin{equation}
		\alpha({\bf z}_n) = p({\bf x}_n\vert {\bf z}_n) \int_{{\bf z}_{n-1}} \alpha({\bf z}_{n-1})p({\bf z}_n\vert{\bf z}_{n-1}) d{\bf z}_{n-1}
	\end{equation}
\end{proposition}

\begin{proof}
	\texttt{to-do: write proof}
\end{proof}

%  	* We show that beta can be represented as a recursive formula
\begin{proposition}
	$\beta({\bf z}_n)$ can be written recursively as
	\begin{equation}
		\beta({\bf z}_n) = \int_{{\bf z}_{n+1}} \beta({\bf z}_{n+1})p({\bf x}_{n+1}\vert {\bf z}_{n+1}) p({\bf z}_{n+1}\vert {\bf z}_n) d{\bf z}_{n+1}
	\end{equation} 
\end{proposition}

\begin{proof}
	\texttt{to-do: write proof}
\end{proof}

% We argue that alpha and beta values can become small very quick, so we require another way to compute its values
For moderately large $N$, $\alpha$ and $\beta$ become small very quick. To solve this we work with re-scaled values for $\alpha$ and $\beta$. This has the additional benefit that the re-scaled values $\alpha({\bf z}_n)$ can be written in form of a Normal distribution whose updating equations are  called the \textbf{Kalman equations}.

% Introduce alpha hat and beta hat: show that they can be written

\begin{proposition}
	Defining the scaled value $\hat\alpha({\bf z}_n)$ as
	\begin{equation}
		\hat\alpha({\bf z}_n) := \frac{\alpha({\bf z}_n)}{p({\bf x}_1, \ldots, {\bf x}_n)} = p({\bf z}_n \vert {\bf x}_1, \ldots, {\bf x}_n)
	\end{equation}
	
	results in an updating equation of the form
	
	\begin{equation}
		\hat\alpha({\bf z}_n) c_n = p({\bf x}_n\vert{\bf z}_n)\int \hat\alpha({\bf z}_{n-1})p({\bf z}_n \vert {\bf z}_{n-1}) d{\bf z}_{n-1}
	\end{equation}
	
	with
	
	\begin{equation}
		c_n = \int_{{\bf z}_n} p({\bf x}_n \vert {\bf z}_n) \int_{{\bf z}_{n-1}} \hat\alpha({\bf z}_{n-1})p({\bf z}_n \vert {\bf z}_{n-1}) d{\bf z}_n{\bf z}_{n-1}
	\end{equation}
\end{proposition}

\begin{proof}
	\texttt{to-do: write proof}
\end{proof}

\begin{proposition}
	Defining the scaled value $\hat\beta({\bf z}_n)$ as
	\begin{equation}
		\hat\beta({\bf z}_n) = \frac{p({\bf x}_{n+1}, \ldots, {\bf x}_N \vert {\bf z}_n)}{p({\bf x}_{n+1}, \ldots, {\bf x}_N \vert {\bf x}_1, \ldots {\bf x}_n)}
	\end{equation}
	results in an updating equation of the form
	\begin{equation}
		\hat\beta({\bf z}_n) = \frac{1}{c_{n+1}}\int p({\bf x}_{n+1}\vert{\bf z}_{n+1})p({\bf z}_{n+1}\vert{\bf z}_n) d{\bf z}_{n+1}
	\end{equation}
\end{proposition}

\begin{proof}
	\texttt{to-do: write proof}
\end{proof}

The term $\hat\alpha({\bf z}_n)$ is called the $\alpha$-forward message passing for a linear dynamical system or \textbf{Kalman filter equation}.  The term $\hat\beta({\bf z}_n)$ is called the $\beta$-backward message passing of a linear dynamical system or \textbf{Kalman smoother equation}. Intuitively, $\hat\alpha({\bf z}_n)$ represents the information that the history of the data has on the $n$-th observation, whereas $\hat\beta({\bf z}_n)$ represents how a latent variable with with known value at time $n$ affects the future behaviour of the system.

As we have previously noted, the term $\hat\alpha({\bf z}_n)$ can be represented as the probability density function of a normal distribution whose mean and covariance function that depends on previous covariance functions. We formalise this in the following two propositions

% to-do: write it more clearly
\begin{proposition}
	The factor $\hat\alpha({\bf z}_1)$ can be written as normal probability density function with mean and covariance matrix given by
	\begin{align}
		\boldsymbol{\mu}_1 &= \boldsymbol{\mu}_0 + {\bf K}_1({\bf x}_1 -{\bf C}\boldsymbol\mu_0)\\
		{\bf V}_1 &=  ({\bf I} - {\bf K}_1{\bf C}){\bf V}_0
	\end{align}
	and normalisation coefficient
	\begin{equation}
		c_1 = \mathcal{N}({\bf x}_1\vert {\bf C}\boldsymbol\mu_0, \boldsymbol\Sigma + {\bf C} {\bf V}_0 {\bf C})
	\end{equation}
	where ${\bf K}_1 = {\bf V}_0{\bf C}({\bf C} {\bf V}_0 {\bf C} + \boldsymbol\Sigma)^{-1}$
\end{proposition}

\begin{proof}
	\texttt{to-do: write proof}
\end{proof}

More generally we have the following proposition

\begin{proposition}
	For every $n \geq 2$, scaled factor $\hat\alpha({\bf z}_n)$ can be written as a normal probability density function with mean and covariance matrix given by
	\begin{align}
		\boldsymbol{\mu}_n &= \boldsymbol{\mu}_{n-1} + {\bf K}_n({\bf x}_n -{\bf C}\boldsymbol\mu_{n-1})\\
		{\bf V}_n &=  ({\bf I} - {\bf K}_n{\bf C}){\bf P}_{n-1}
	\end{align}
	
	where we have defined
	\begin{align}
		{\bf P}_{n-1} &:= \boldsymbol{\Gamma} + {\bf A}{\bf V}_{n-1}{\bf A}^T\\
		{\bf K}_n &:= {\bf P}_{n-1}{\bf C}^T({\bf C} {\bf P}_{n-1}{\bf C}^T + \boldsymbol\Sigma)^{-1}
	\end{align}
\end{proposition}

\begin{proof}
	\texttt{to-do: write proof}
\end{proof}


% Show that alpha hat is a normal distribution
%	* Derive the Kalman Filter Equations
% Show that gamma is also a normal distribution

\begin{proposition}
	$\gamma({\bf z}_n)$ can be written in terms of the re-scaled factors $\hat\alpha({\bf z}_n)$, and $\hat\beta({\bf z}_n)$ as
	\begin{equation}
		\gamma({\bf z}_n) = \hat\alpha({\bf z}_n)\hat\beta({\bf z}_n).
	\end{equation}
\end{proposition}

\begin{proof}
	\texttt{to-do: write proof}
\end{proof}

\begin{proposition}
	$\xi({\bf z}_{n-1}, {\bf z}_n)$ can be written in terms of the re-scaled factors $\hat\alpha({\bf z}_n)$, and $\hat\beta({\bf z}_n)$ as
	\begin{equation}
		\xi({\bf z}_{n-1}, {\bf z}_{n}) = c_n^{-1}\hat\alpha({\bf z}_n)p({\bf x}_n \vert {\bf z}_n) p({\bf z}_{n-1}\vert {\bf z}_n)\hat\beta({\bf z}_n)
	\end{equation}
\end{proposition}

\begin{proof}
	\texttt{to-do: write proof}
\end{proof}

\subsection{The M-step}
...

\section{Graphical Models}
In this section we present some useful proposition of graphical models.

\begin{proposition}\label{prop:graphical-models-separation}
	Let ${\bf Z} = \{{\bf z}_n\}_n$ be a set of latent random variables, and ${\bf X} = \{{\bf x}_n\}_n$ a set of observed variables with complete-data likelihood $({\bf Z}, {\bf X})$ given by the LDS model. Then, the following factorisations hold true
	\begin{align}
		p({\bf X}\vert{\bf z}_n) &= p({{\bf x}_1, \ldots, {\bf x}_n \vert {\bf z}_n})p({{\bf x}_{n+1}, \ldots, {\bf x}_N \vert {\bf z}_n}) \label{gm:1}\\
		p({\bf x}_1, \ldots, {\bf x}_{n-1}\vert {\bf x}_n, {\bf z}_n) &= p({\bf x}_1, \ldots, {\bf x}_{n-1}\vert {\bf z}_n)\\
		p({\bf x}_1, \ldots, {\bf x}_{n-1}\vert {\bf z}_{n-1}, {\bf z}_{n}) &= p({\bf x}_1, \ldots, {\bf x}_{n-1}\vert {\bf z}_{n-1})\\
		p({\bf x}_{n+1}, \ldots, {\bf x}_N \vert {\bf z}_n, {\bf z}_{n+1}) &= p({\bf x}_{n+1}, \ldots, {\bf x}_N \vert {\bf z}_n)\\
		p({\bf x}_{n+2}, \ldots, {\bf x}_N\vert {\bf x}_{n+1}, {\bf z}_{n+1}) &= p({\bf x}_{n+2}, \ldots, {\bf x}_N \vert {\bf z}_{n+1})\\
		p({\bf X}\vert {\bf z}_{n-1}, {\bf z}_{n}) &= p({\bf x}_1, \ldots, {\bf x}_{n-1}\vert {\bf z}_{n-1}) \nonumber\\
			&\hspace{1cm}p({\bf x}_n\vert {\bf z}_n) p({\bf x}_{n+1}, \ldots, {\bf x}_N \vert {\bf z}_n)\\
		p({\bf x}_{N+1} \vert {\bf X}, {\bf z}_{N+1}) &= p({\bf x}_{N+1} \vert {\bf z}_{N+1})\\
		p({\bf z}_{N+1} \vert {\bf X}, {\bf z}_{N}) &= p({\bf z}_{N+1} \vert {\bf z}_{N})\\
	\end{align}
\end{proposition}

\begin{proof}
	\texttt{to-do: show by D-separation or explicitly?}
\end{proof}

\end{document}