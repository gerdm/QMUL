\documentclass[11pt]{article}
\usepackage[utf8]{inputenc}
\usepackage{enumitem}

%\usepackage{biblatex}
%\addbibresource{bib.bib}

\usepackage{natbib}
\usepackage{bibentry}
\bibliographystyle{plainnat}


\title{MTH700 Assesed Coursework}
\author{Gerardo Durán Martín}
\date{November 2020}

\begin{document}
\nobibliography{bib}
\maketitle

\section*{Section A: Literature Research}
The concept of an \textit{iterated function system} as a means of constructing fractal sets was introduced in the 1980s.
\begin{enumerate}[label=(\alph*)]
	\item Provide a complete reference to the paper where this concept was introduced \\ \textbf{Ans}:
	\begin{itemize}
	\item \bibentry{itf-ref0}
	\end{itemize}
	\item How many citations has this paper received so far? \\ \textbf{Ans}: As of November 2020, the the paper has received 255 citations so far, according to Google Scholar. However, the page where the paper is being cited from, ACM, estimated 107 total citations. 
	\item Provide complete references of \textit{two} papers published in the journal \textit{Inventiones Mathematicae} in the 2010s which make use of the concept of an iterated function system. \\ \textbf{Ans}:
	\begin{itemize}
		\item \bibentry{itf-ref1}
		\item \bibentry{itf-ref2}
	\end{itemize}
	\item Provide a reference to a source which one might consult for an accessible introduction to this concept \\ \textbf{Ans}:
	\begin{itemize}
		\item \bibentry{itf-ref3}
	\end{itemize}
\end{enumerate}

\section*{Section B: Literature Review}
\subsection*{Paper 1}
D. A. Goldston, J. Pintz, C. Y. Yıldırım. Primes in Tuples I, Annals of Mathematics, 170, 819–862, 2009.
\begin{enumerate}[label=(\alph*)]
	\item What is the main message of the paper? \\ \textbf{Ans}: Introduce a method presumed to help in proving the existence of infinitely many prime tuples.
	\item Why is the paper relevant? \\ \textbf{Ans}: They provided a new way to think about the twin prime conjecture, which was later later on used to prove that there exists an infinite set of two consecutive prime numbers that are below some fixed gap.
	\item Summary of the paper \\ \textbf{Ans}: The paper introduces three new theorems that show what was possible to know regarding gaps between consecutive prime numbers and the limitations of current methods.\\ Their first theorem follows from a conjecture which, if proven to be true, would mean that there are an infinite amount of consecutive prime numbers whose distance differ by, at most, 16.\\ Their second theorem shows that there are an infinite amount of consecutive prime numbers whose distance differ by no more than the log of the smallest of the pair.\\ Their final theorem places an upper bound for the distance between two prime numbers separated by two or more numbers.
\end{enumerate}

\subsection*{Paper 2}
Liu, Yang-Yu, Jean-Jacques Slotine, and Albert-László Barabási. Controllability of complex networks. Nature 473, 167-173, 2011.
\begin{enumerate}[label=(\alph*)]
	\item What is the main message of the paper? \\ \textbf{Ans}: One can make use of analytical tools to study which nodes guide the dynamics (behaviour) of an arbitrary complex directed networks. This requires to know where in a complex network should one place \textit{input} nodes to drive the state of the network to a different configuration. 
	\item Why is the paper relevant? \\ \textbf{Ans}: They were able to show that the minimum number of nodes that control the dynamics of a complex network is given by the degree distribution, i.e., the probability distribution of how many connections a given node contains.
	\item Summary of the paper \\ \textbf{Ans}: The paper introduces a new method to understand which nodes in a complex directed network drive the dynamics of the network. The authors divided their work in four sections. \\ In the first section, the authors lay out the basic concepts of controllability and show that the Kalman's controllability rank condition is sufficient to identify the minimum number of nodes required to control the network.\\ In the next section, they make use of their method to perform experiments. They find, for instance, that in social networks, few individuals may be able to control the system.\\Then, they make use of the minimum number of nodes to show the role that plays the \textit{density} of the of the network.\\Finally, they analyze how robust the controllability of a network is in terms of the mean degree of each network.
\end{enumerate}

\subsection*{A critical review of the ``Controllability of Complex Networks'' paper}
\begin{itemize}
	\item Summary of the main message
	\item Assessment of its message and its methods
	\item A discussion on its readability (motivation, logical coherence, simplicity)   
\end{itemize}

In the paper, The Controllability of Complex Networks, the authors show that by selecting a number of input nodes --which are determined by the architecture of the network-- one can drive the state of the whole network by manipulating only the input nodes. More specifically, they show that the degree distribution of the network is enough to know the minimum number of \textit{input} nodes that a network should have in order to control the state of the system.

The idea of being able to control the nodes in a network which drive the system as a whole is a fascinating idea. One might think that in a network, the hubs are the responsible to control how the system behaves, however, their research shows the exact oposite. It stroke me as interesting to know that the rank of the matrix of connecting and input nodes is enough to understand whether the system is controllable.

Regarding the readability. I do not think this read is for 

\section*{Section C: Communicating Mathematics}

\end{document}