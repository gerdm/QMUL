\documentclass[11pt]{article}
\usepackage[utf8]{inputenc}
\usepackage{enumitem}

%\usepackage{biblatex}
%\addbibresource{bib.bib}

\usepackage{natbib}
\usepackage{bibentry}
\bibliographystyle{plainnat}


\title{MTH700 Assesed Coursework}
\author{Gerardo Durán Martín}
\date{November 2020}

\begin{document}
\nobibliography{bib}
\maketitle

\section*{Section A: Literature Research}
The concept of an \textit{iterated function system} as a means of constructing fractal sets was introduced in the 1980s.
\begin{enumerate}[label=(\alph*)]
	\item Provide a complete reference to the paper where this concept was introduced \\ \textbf{Ans}:
	\begin{itemize}
	\item \bibentry{itf-ref0}
	\end{itemize}
	\item How many citations has this paper received so far? \\ \textbf{Ans}: As of November 2020, the the paper has received 255 citations so far, according to Google Scholar. However, the page where the paper is being cited from, ACM, estimated 107 total citations. 
	\item Provide complete references of \textit{two} papers published in the journal \textit{Inventiones Mathematicae} in the 2010s which make use of the concept of an iterated function system. \\ \textbf{Ans}:
	\begin{itemize}
		\item \bibentry{itf-ref1}
		\item \bibentry{itf-ref2}
	\end{itemize}
	\item Provide a reference to a source which one might consult for an accessible introduction to this concept \\ \textbf{Ans}:
	\begin{itemize}
		\item \bibentry{itf-ref3}
	\end{itemize}
\end{enumerate}

\section*{Section B: Literature Review}
\subsection*{Paper 1}
D. A. Goldston, J. Pintz, C. Y. Yıldırım. Primes in Tuples I, Annals of Mathematics, 170, 819–862, 2009.
\begin{enumerate}[label=(\alph*)]
	\item What is the main message of the paper? \\ \textbf{Ans}: Introduce a method presumed to help in proving the existence of infinitely many prime tuples.
	\item Why is the paper relevant? \\ \textbf{Ans}: They provided a new way to think about the twin prime conjecture, which was later later on used to prove that there exists an infinite set of two consecutive prime numbers that are below some fixed gap.
	\item Summary of the paper \\ \textbf{Ans}: The paper introduces three new theorems that show what was possible to know regarding gaps between consecutive prime numbers and the limitations of current methods.\\ Their first theorem follows from a conjecture which, if proven to be true, would mean that there are an infinite amount of consecutive prime numbers whose distance differ by, at most, 16.\\ Their second theorem shows that there are an infinite amount of consecutive prime numbers whose distance differ by no more than the log of the smallest of the pair.\\ Their final theorem places an upper bound for the distance between two prime numbers separated by two or more numbers.
\end{enumerate}

\subsection*{Paper 2}
Liu, Yang-Yu, Jean-Jacques Slotine, and Albert-László Barabási. Controllability of complex networks. Nature 473, 167-173, 2011.
\begin{enumerate}[label=(\alph*)]
	\item What is the main message of the paper? \\ \textbf{Ans}: To show that in a complex network, the driver nodes, which can guide the entire dynamics of the system, is determined by the network's degree distribution.
	\item Why is the paper relevant? \\ \textbf{Ans}:
	\item Summary of the paper \\ \textbf{Ans}:
\end{enumerate}

\section*{Section C: Communicating Mathematics}

\end{document}